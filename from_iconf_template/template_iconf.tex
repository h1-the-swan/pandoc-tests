%%%%%%%%%%%%%%%%%%%%%%%%%%%%%%%%%%%%%%%%%
% iConference Article
% LaTeX Template
% Version 2.1 (25/10/13)
%
% This template has been downloaded from:
% http://www.LaTeXTemplates.com
%
% Original author:
% Frits Wenneker (http://www.howtotex.com)
%
% Modified by:
% Joseph Helsing (2012)
% Heinz-Alexander Fütterer, Maxi Kindling, Stephanie van de Sandt (2013)
% Cory Knobel (2014)
%
% License:
% CC BY-NC-SA 3.0 (http://creativecommons.org/licenses/by-nc-sa/3.0/)
%
%%%%%%%%%%%%%%%%%%%%%%%%%%%%%%%%%%%%%%%%%

%----------------------------------------------------------------------------------------
%	LIST OF NECESSARY PACKAGES
%----------------------------------------------------------------------------------------

% The packages used in this template were: fontenc, babel, microtype, indentfirst,
% geometry, multicol, graphicx, apacite, caption, booktabs, float, footnote, paralist,
% titlesec, fancyhdr, xcolor, tabularx, lipsum

%----------------------------------------------------------------------------------------
%	PACKAGES AND OTHER DOCUMENT CONFIGURATIONS
%----------------------------------------------------------------------------------------
\documentclass[10pt, letterpaper]{article}
\usepackage{lipsum} % Package to generate dummy text throughout this template
\usepackage[T1]{fontenc} % Use 8-bit encoding that has 256 glyphs
\usepackage{lmodern}
\usepackage[english]{babel}
\linespread{1.05} % Line spacing - Palatino needs more space between lines
\usepackage{microtype} % Slightly tweak font spacing for aesthetics
\setlength{\parindent}{0.5in} % Indent 0,5 inch
\usepackage[top=.8in, bottom=0.8in, left=1in, right=1in]{geometry} % Document margins
\usepackage{graphicx}
\usepackage{xcolor}
\usepackage[implicit=false,colorlinks=true,allcolors={black},urlcolor={black}]{hyperref}

\usepackage{apacite} % Allows for the correct APA citation of references
\renewcommand{\bibliographytypesize}{\fontsize{10pt}{10pt}} % Set the bibliography font size to 10pt

\usepackage{url}
\usepackage{tabularx}
\usepackage{booktabs} % Horizontal rules in tables
\usepackage[sf]{titlesec} % Font of Headings

\usepackage[normal]{caption} % Custom captions under/above floats in tables or figures
\captionsetup{format=plain,justification=RaggedRight,singlelinecheck=false}
\usepackage{footnote} %Enables footnotes

\makeatletter
\def\blfootnote{\selectfont\xdef\@thefnmark{}\@footnotetext}
\makeatother %This is used to allow for blank footnotes

\usepackage{paralist} % Used for the compactitem environment which makes bullet points with less space between them

\usepackage{fancyhdr} % Headers and footers
\pagestyle{fancy} % All pages have headers and footers
\fancyhf{}
\fancyhead[L]{\vspace{-15mm}\fontsize{10pt}{10pt}\selectfont{iConference 2016} }%This is the header
\fancyhead[R]{\vspace{-15mm}\fontsize{10pt}{10pt}\selectfont{[Name of the first author] et. al (if two, write both)} }
\fancyfoot[C]{\thepage} % Custom footer text

\usepackage{tocloft} % Change List of Figures / Tables
\renewcommand{\cftfigpresnum}{Figure }
\setlength{\cftfignumwidth}{5em}
\renewcommand{\cfttabpresnum}{Table }
\setlength{\cfttabnumwidth}{5em}
\renewcommand\cftloftitlefont{\sffamily \Large}
\renewcommand\cftlottitlefont{\sffamily \Large}

%----------------------------------------------------------------------------------------
%	TITLE SECTION
%----------------------------------------------------------------------------------------

\title{$title$}
%----------------------------------------------------------------------------------------

\date{} % no date

% Formatting title and authors.
\makeatletter
\renewcommand{\maketitle}{\bgroup\setlength{\parindent}{0pt}
\begin{flushleft}
  {\sffamily \Large {\@title} }
  \vspace{12pt}\\
  \@author
\end{flushleft}\egroup
}
\makeatother

%----------------------------------------------------------------------------------------
% AUTHORS
%----------------------------------------------------------------------------------------

\author{$for(author)$$author.name$$sep$, $endfor$}

%----------------------------------------------------------------------------------------

\begin{document}

\newenvironment{blockquote}{\list{}{\leftmargin=0.5in\rightmargin=0.0in}\item[]}{\endlist}
\renewcommand{\listfigurename}{Table of Figures} 
\renewcommand{\listtablename}{Table of Tables} 
\maketitle % Insert title
\thispagestyle{empty} % First page has no header,footer

%----------------------------------------------------------------------------------------
%	INFOBOX
%----------------------------------------------------------------------------------------

\begin{center}
\begin{tabularx}{\textwidth}{|X|}
\hline
\vspace{2pt}\\
\textbf{Abstract}\\
The abstract shall be no more than 150 words.  All work must be in English.\\
{\footnotesize \textbf{Keywords:} Enter up to five keywords separated by a semicolon (keyword1; keyword 2; keyword3; keyword4; keyword5)}\\
{\footnotesize \textbf{Citation:}  Editor will add citation.}\\
{\footnotesize \textbf{Copyright:} Copyright is held by the authors.}\\
{\footnotesize \textbf{Acknowledgements:} If there is an acknowledgement add it here. This field is \em{optional}. For review purpose, do not add information to this field until after the paper is accepted.}\\ % Please delete line if not used.
{\footnotesize \textbf{Research Data:}  If there is any research data / raw data contact the editor. The editor will add research data information such as a DOI. This field is \em{optional}.} \\ % Please delete line if not used.
{\footnotesize \textbf{Contact:} Author will add e-mail address.}\\ % Please enter at least one contact e-mail address. 
\vspace{2pt}\\
\hline
\end{tabularx}
\end{center}

%----------------------------------------------------------------------------------------
%	ARTICLE CONTENTS
%----------------------------------------------------------------------------------------
$body$
%----------------------------------------------------------------------------------------
%	TABLE EXAMPLE
%----------------------------------------------------------------------------------------
\begin{table}[h]

\begin{tabular}{lllll} %This sets the number of columns, one l per column
\toprule
Head & Head & Head & Head & Head \\
\midrule
Label & 000* & 000 & 000 & 000 \\
Label & 000 & 000* & 000 & 000 \\
Label\textsuperscript{a} & 000 & 000 & 000* & 000*\\
\bottomrule
\end{tabular}
\caption{Title of Table} %This sets the table's title
\end{table}
%----------------------------------------------------------------------------------------
%	END TABLE EXAMPLE
%----------------------------------------------------------------------------------------

%----------------------------------------------------------------------------------------
%	FIGURE EXAMPLE
%----------------------------------------------------------------------------------------

%Make sure the figure file that you want to add is in the same folder as this .tex file or it may not show up when compiled.


%----------------------------------------------------------------------------------------
%	END FIGURE EXAMPLE
%----------------------------------------------------------------------------------------

\section{Heading}

%----------------------------------------------------------------------------------------
%	LIST EXAMPLE
%----------------------------------------------------------------------------------------

\begin{itemize}
\item This
\item is
\item a
\item list.
\end{itemize}

\begin{enumerate}[a.]
\item This
\item is
\item a
\item list.
\end{enumerate}

%----------------------------------------------------------------------------------------
%	END LIST EXAMPLE
%----------------------------------------------------------------------------------------



%----------------------------------------------------------------------------------------
%	BIBLIOGRAPHY
%----------------------------------------------------------------------------------------
\fontsize{10pt}{10pt}\selectfont
\urlstyle{same} %Sets the URL fonts to be the same as the current font.
\bibliographystyle{apacite}
\bibliography{confbib} % You can use another bibtex file if you desire, or you can just remove the contents of confbib and add in your own entries.


\end{document}
